\title{Individual differences{\strut} in early instructed language learning}
\subtitle{The role of language aptitude, cognition, and motivation}
\author{Raphael Berthele and Isabelle Udry}
\renewcommand{\lsSeries}{eurosla}
\renewcommand{\lsSeriesNumber}{}
\lsCoverTitleSizes{44pt}{19pt}% Font setting for the title page

\renewcommand{\lsID}{313}

\BackBody{Variability in predispositions for language learning has attracted
scholarly curiosity for over 100 years. Despite major changes in
theoretical explanations and foreign/second language teaching paradigms,
some patterns of associations between predispositions and learning
outcomes seem timelessly robust. This book discusses evidence from a
research project investigating individual differences in a wide variety
of domains, ranging from language aptitude over general cognitive
abilities to motivational and other affective and social constructs. The
focus lies on young learners aged 10 to 12, a less frequently
investigated age in aptitude research. The data stem from two samples of
multilingual learners in German-speaking Switzerland. The target
languages are French and English. The chapters of the book offer two
complementary perspectives on the topic: on the one hand,
cross-sectional investigations of the underlying structure of these
individual differences and their association with the target languages
are discussed. Drawing on factor analytical and multivariable analyses,
the different components are scrutinized with respect to their mutual
dependence and their relative impact on target language skills. The
analyses also take into account contextual factors such as the learners’
family background and differences across the two contexts investigated.
On the other hand, the potential to predict learner’s skills in the
target language over time based on the many different indicators is
investigated using machine learning algorithms.
The results provide new insights into the stability of the individual
dispositions, on the impact of contextual variables, and on empirically
robust dimensions within the array of variables tested.}
