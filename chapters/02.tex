\documentclass[output=paper]{langsci/langscibook} 
\ChapterDOI{10.5281/zenodo.5464747}
\author{Carina Steiner\orcid{}\affiliation{University of Berne, Center for the Study of Language and Society} and Raphael Berthele\orcid{}\affiliation{University of Fribourg, Institut de Plurilinguisme} and Isabelle Udry\orcid{}\affiliation{University of Fribourg, Institut de Plurilinguisme; Zurich University of Teacher Education}}
\title{Language Aptitude at Primary School (LAPS): Research design}
\abstract{This chapter delineates the design of the LAPS project. We start with an outline of the research questions, followed by a description of the curricular context of foreign language learning at Swiss primary schools. Next, we will give a full description of the test instruments and how they were implemented, as well as details on the participants and procedures. Finally, data entry and scoring will be outlined.}
\IfFileExists{../localcommands.tex}{
  \addbibresource{localbibliography.bib}
  % add all extra packages you need to load to this file

\usepackage{tabularx,multicol}
\usepackage{url}
\urlstyle{same}

\usepackage{listings}
\lstset{basicstyle=\ttfamily,tabsize=2,breaklines=true}

%\usepackage{langsci-optional}
\usepackage{langsci-lgr}
\usepackage{langsci-gb4e}
\usepackage{langsci-optional}

\usepackage{enumitem}
\usepackage[group-digits=false, detect-weight=true]{siunitx}

\usepackage{todonotes}

  \newcommand*{\orcid}{}
 
  %% hyphenation points for line breaks
%% Normally, automatic hyphenation in LaTeX is very good
%% If a word is mis-hyphenated, add it to this file
%%
%% add information to TeX file before \begin{document} with:
%% %% hyphenation points for line breaks
%% Normally, automatic hyphenation in LaTeX is very good
%% If a word is mis-hyphenated, add it to this file
%%
%% add information to TeX file before \begin{document} with:
%% %% hyphenation points for line breaks
%% Normally, automatic hyphenation in LaTeX is very good
%% If a word is mis-hyphenated, add it to this file
%%
%% add information to TeX file before \begin{document} with:
%% \include{localhyphenation}
\hyphenation{
affri-ca-te
affri-ca-tes 
Soa-res
scru-ti-ny
me-ta-cog-ni-tion
}

\hyphenation{
affri-ca-te
affri-ca-tes 
Soa-res
scru-ti-ny
me-ta-cog-ni-tion
}

\hyphenation{
affri-ca-te
affri-ca-tes 
Soa-res
scru-ti-ny
me-ta-cog-ni-tion
}
 
  \togglepaper[1]%%chapternumber
}{}

\begin{document}
\maketitle 
%\shorttitlerunninghead{}%%use this for an abridged title in the page headers







\section{Research questions}

The aim of the project \textit{Language Aptitude at Primary School} (LAPS) is to explore the extent to which skills, abilities, and socio-environmental factors contribute to successful language learning by primary school children. 

We consider individual difference (ID) variables and environmental factors previously found to affect foreign language learning along four broad categories:

\begin{enumerate}\sloppy
\item Language aptitude according to \citet{Carroll1958}: Phonetic coding ability, grammatical sensitivity, inductive ability, rote memory;
\item General cognitive abilities or general learning abilities: Intelligence, working memory, creativity, field independence;
\item Affective dispositions: L2/L3 motivation, L2/L3 self-concepts, foreign language learning anxiety, dedication, perceived teacher encouragement, perceived parental encouragement, locus of control;
\item Environmental factors: Socio-economic status (SES), language background, region, teaching paradigm (i.e. task-based teaching and learning as prescribed by the Swiss curriculum).
\end{enumerate}

The theoretical underpinnings of these variables are discussed in Chapter 1. In the LAPS project, we address the following research questions: 

\begin{enumerate}
\item What are the underlying dimensions of the ID variables assessed in the test battery and how do these dimensions relate to L2 proficiency? (Chapter 3)
\item What is the predictive value of the ID variables and environmental factors for young learners’ L2 proficiency? (Chapter 4)
\item What is the relationship between the ID variables, environmental factors, and foreign language skills?
    \begin{itemize}
    \item Association of socio-environmental background variables with foreign language ability (Chapter 5)
    \item Creativity and the task-based learning environment (Chapter 6)
    \item Affective predispositions and the proximity to the target language community (Chapter 7)
    \end{itemize}
\item What are the developmental patterns identified over two academic years (1.5 years of foreign language learning) in terms of:
    \begin{itemize}
    \item The dynamics of affective dispositions (Chapter 8)
    \item Competences in school language German and L2 English (Chapter 9)
    \item The language analysis component of language aptitude, i.e. grammatical sensitivity and inductive ability (Chapter 10)
    \end{itemize}
\end{enumerate}

\section{Study context}\label{sec:02:2}

The research questions were investigated in two subprojects between 2017 and 2019.  LAPS I took place in a German-speaking region close to the French-German language border. The children learnt L2 French and L3 English. It was designed as a cross-sectional study which was complemented by a second data collection to further investigate affective dispositions and L3 proficiency.  LAPS I also served as a pilot study for the test battery. 

LAPS II was conducted in the north-eastern German-speaking part of Switzerland. Participants learnt L2 English and L3 French. It was longitudinal with three data collections (T1, T2, T3) over two academic years. We followed the development of L2 English proficiency, L1 school language proficiency in German, aptitude (language analysis component), and affective dispositions. 

 \subsection{The foreign language curriculum in Switzerland}\label{sec:02:2.1}


All Swiss children learn two foreign languages as part of the mandatory curriculum: English and one national language, i.e. French, Italian, or Rumantsch. The cantons (equivalent to provinces or districts) are free to organise their own language curricula, although, based on a constitutional article accepted by the people in 2006, the last decades were characterized by a tendency to harmonize curricula across the country. Despite this tendency, cantons are still free to choose which two languages they want children to learn and in what order. Overall, this has resulted in two different systems across the German-speaking part of Switzerland where this study took place:

\begin{enumerate}[label=\alph*.]
\item Regions close to the French-speaking area introduce French in 3\textsuperscript{rd} grade and English in 5\textsuperscript{th} grade. Similarly, regions close to the Italian and Rumantsch speaking parts of the country choose Italian and/or Rumantsch as one of their languages.
\item The other regions start with English as the first foreign language in 3\textsuperscript{rd} grade and introduce a national language later, e.g. French, in 5\textsuperscript{th} grade.\footnote{The region where LAPS II was conducted is an exception: Until 2019 L2 English was introduced in 2\textsuperscript{nd} grade. In 2020 the region adapted to the common national practice, i.e. L2 English classes now start in 3\textsuperscript{rd} grade.}
\end{enumerate}

As mentioned before, in the current project, both systems are represented. LAPS I took place in region a) with L2 French and L3 English, and LAPS II in region b) with L2 English and L3 French.

Each foreign language is taught for 2 to 3 lessons a week, depending on the grade. Tables~\ref{tab:02:2} and~\ref{tab:02:5} indicate the total number of lessons children in the LAPS project had attended at different times of testing. 

\subsection{Overall goals of the Swiss foreign language curriculum}
Foreign language teaching in Switzerland aims at developing functional multilingualism. All four skills are taught in the L2 and L3 from the start: Listening, reading, speaking, and writing. The curriculum also contains a domain called \textit{Sprachen im Fokus} (languages in focus) which covers formal aspects of language (including grammar and pronunciation), language awareness, and the use of strategies. \textit{Kulturen im Fokus} (culture in focus) stipulates objectives of cultural knowledge and attitudes. In keeping with communicative approaches to language teaching and learning, fluency is given priority over accuracy. At the end of primary school (age 12), children should reach beginner levels in the L2 and L3. More specifically, the national standards based on the CEFRL\footnote{Common European Framework of Reference for Languages} target L2 levels of A2.1 in listening, reading and speaking, and A1.2 in writing. In the L3, children are expected to reach A1.2 levels in all areas of competence \citep[17]{Zuerich2017}. 

Children are taught in a task-supported approach, i.e. a weak form of communicative task-based learning and teaching \citep{Ellis2017}. This means that tasks are central to the lesson, but they are usually complemented with form-focused elements. As described in more detail in Chapter 6, this method is reflected in the teaching manuals which in most cantons are prescribed by the local board of education. The teaching manuals contain several units structured around a topic that is introduced via authentic input followed by meaning-focused activities. Vocabulary learning and some elements of explicit grammar are also part of the lesson plans. At the end of each unit, learners use all aspects of language they have acquired to complete a communicative task about the topic, such as writing a poem, or doing role plays. 

The Swiss curriculum prescribes awareness raising elements that draw on all languages in a student’s repertoire (\citealt{EDK2004}, \citealt{Passepartout2008}). The aim is to enhance learning under relatively limited input conditions by developing metalinguistic awareness. Therefore, teaching manuals include specific sections on intercomprehension and language learning strategies that should encourage transfer among languages. However, these sections are not the main focus of the manuals and teachers therefore integrate them flexibly into their lessons \citep[8]{Zuerich2017}. 

\section{Test battery}\label{sec:02:3}

The ID variables were assessed in a comprehensive test battery, including psychometric tests and two student questionnaires (motivation, locus of control) and a parent questionnaire (family background information). Where possible, the constructs were measured with standardized tools. However, some tests had to be translated into German and/or adapted to the age of our participants. In the following, short descriptions of each test are given. Tables~\ref{tab:tests:part-1}--\ref{tab:tests:part-4} summarize dimensions, conditions for administration, and in which subproject the tests were used. The reliability analysis is available in the technical report at \url{https://osf.io/hstv7/}.

The test battery was trialled in LAPS I with 10 classes of 4\textsuperscript{th} and 5\textsuperscript{th} graders from 9 different schools. Some changes were added in consultation with the scientific advisory board of the project before LAPS II (see 3.2).

\subsection{Tests for language aptitude}

\begin{description}\sloppy
\item[Team up Words!] Test of grammatical sensitivity based on the MLAT-E, part 2 Matching Words (\citealt{CarrollSapon2010}), translated into German and adapted for the target group of the present study.

Participants are instructed to identify functions of words in sentences (no explicit grammatical terms are used). After the training phase, they are presented with paired sentences. In the first sentence, one word is highlighted. The participants’ task is to find the corresponding word (i.e. the word with the same grammatical function) in the second sentence.

\item[Language Detective] Test for inductive abilities based on PLAB form 4 \citep{PimsleurEtAl2004}, translated and adapted for the target group of the present study.

Participants are presented with a list of words and short sentences in an artificial language as well as their translation in German (=language of instruction). From this input, participants have to deduce how sentences in the artificial language may be formed.

\item[Llama-E] Test for sound-symbol association and phonemic working memory \citep{MearaEtAl2001}. In the training phase, participants learn how sounds are represented graphically in an artificial language. In the test phase, they listen to bi-syllabic words in the artificial language and have to choose the correct “spelling” between two options.

\item[Llama-D] Sound recognition task \citep{MearaEtAl2001}. In the training phase, participants hear different sounds of an artificial language. In the test phase, participants listen to strings of sounds (combined set of training and novel sounds) and are asked to identify the sounds they have already heard in the training.
\end{description}

\subsection{Tests for cognition/general learning abilities}\largerpage
\begin{description}\sloppy
\item[CFT 20-R: Matrices and topological deductions {\upshape\citep{Weiss2006}}:]  Fluid intelligence was assessed with two subtests from the CFT 20-R \citep{Weiss2006}. In the matrices subtests, geometric patterns with a missing piece are presented to the participants. They are given five choices to pick from and fill in the missing piece. The topologies deductions subtest consists of geometric pattern containing one to three dots. Participants are asked to select, from five given options, the one that replicates the conditions in the example (e.g. the one where a dot could be placed outside a triangle but inside a circle).

\item[CFT-20-R: Number sequences {\upshape\citep{Weiss2006}}:]  In this test for crystallized intelligence, participants are presented with a sequence of numbers in a certain pattern and are asked to choose the next logical number from five options.

\item[Corsi Blocks:] The Corsi Blocks task assesses visual working memory. An increasingly long sequence of squares on the screen is lit up, and participants are asked to reproduce the correct order.

\item[Digit Span (Forward/Backward):] Number strings of increasing length are presented visually and aurally. The participants have to reproduce them either in the order of presentation (forward) or in reversed order (backward) until they reach maximum recall capacity. The Digit Span task measures verbal working memory. The forward version is a simple measure of short-term phonological memory (or phonological loop), and the Backward Digit Span is a complex task tapping into executive working memory.

\item[Alphabet Task:] A measure of automatic letter access, retrieval, and production by \citet{BerningerEtAl1992} in which participants are instructed to write down the alphabet as quickly as possible without sacrificing legibility. The Alphabet task has been associated with children’s levels of composition in the L1 (\citealt{BerningerEtAl1997}, \citealt{GrahamEtAl2006}).

\item[Test of Creative Thinking (Divergent Production) (TCT-DP) {\upshape\citep{UrbanJellen1995}}:] Participants are presented with a sheet of paper containing a frame with 6 figural fragments (a semicircle, waveline, dot, right angle, dashed line, and a lying “u” located outside the frame). They are asked to complete the unfinished picture without a time limit. The test is scored according to 14 criteria which include boundary breaking, risk taking, introducing new elements, humour, and the ability to link elements to give meaning to the overall picture. Thus, the test goes beyond scoring quantitative aspects of divergent thinking, such as the number of ideas. The test taps more holistically into an individual’s creative potential by considering qualitative aspects of divergent thinking as well, i.e. integrating various elements meaningfully or unexpected ideas.

\item[Group Embedded Figures Test (GEFT):] An assessment of field independence \citep{WitkinEtAl2014} in which participants need to find simple geometrical figures embedded in more complex figures.
\end{description}

\subsection{Assessment of affective dispositions}
\begin{sloppypar}
Based on existing test instruments from \citet{HorwitzEtAl1986}, \citet{Stoeckli2004}, \citet{Doernyei2010}, \citet{Heinzmann2013}, and \citet{PeyerEtAl2016}, we put together a student questionnaire covering the following dimensions: Intrinsic motivation, extrinsic motivation (school/leisure), lingua franca motivation, foreign language learning anxiety, self-concepts (L2\,+\,school language), teacher motivation, parental encouragement, dedication, and future L2 self. The questionnaire comprised a section for L2 and L3 with the same items for each language. 
\end{sloppypar}

Locus of control was assessed with a German translation of the N-S Personality Scale by \citet{NowickiStrickland1973}.

\subsection{Assessment of environmental factors}
A parent questionnaire filled in at the beginning of the study assessed personal and linguistic background (country of origin, years of schooling, L1, family language, literacy language), SES (parents’ highest level of education, n° of books, financial resources, monthly income), and school context (classes in German as a second language\slash heritage language and culture, French/English homework).

\subsection{Tests for language proficiency}
ELFE 1--6, Reading Proficiency (\citealt{LenhardSchneider2006}) is a normed test for reading skills in the language of instruction German. Items are presented at word, sentence, and text level.\largerpage

Oxford Young Learners Placement Test \citep{Testing2013} was used for L2/L3 English. The test consists of two sections: Language use (vocabulary and grammar) and listening (short and extended listening exercises) and is said by the distributors to cover levels A1--B1.

C-tests were used for L2 French and L3 English. Participants need to reconstruct meaning from partly deleted words in a short text, completing the missing part of the words. C-tests measure general language proficiency which is conceptualized as an underlying ability consisting of knowledge and skills displayed in all areas of language use \citep{EckesGrotjahn2006}. The C-tests were based on topics and vocabulary covered in the curriculum. They were piloted with classes who did not participate in the LAPS project. For L2/L3 English, texts were adapted from \citet{BabaiiShahri2010} and \citet{WildenPorsch2017} in accordance with curricular content of the target group.

\begin{table}[p]\footnotesize
\begin{tabularx}{\textwidth}{QQQl}
\lsptoprule
{Subdimension} & {Test} & {Conditions} & {Study}\\\midrule
Grammatical sensitivity & Team up Words! (adapted from MLAT-E subtest Matching Words, \citealt{CarrollSapon2010}) & 30 items, 13min 30s & LAPS I\&II\\
Inductive ability & Language Detective (adapted from PLAB 4, \citealt{PimsleurEtAl2004}) & 15 items, 10min & LAPS I\&II\\
Sound-symbol association (phonemic working memory) & Llama-E \citep{MearaEtAl2001} & Training: 2min., 24 sounds to learn, Test: 20 bi-syllabic words & LAPS I\\
Sound recognition task (phonemic discrimination) & Llama-D \citep{MearaEtAl2001} & Training: 10 words, Test: learned words alongside 20 novel words & LAPS II\\
\lspbottomrule
\end{tabularx}
\caption{Description of language aptitude tests\label{tab:tests:part-1}}
\end{table}

\begin{table}[p]\footnotesize
\begin{tabularx}{\textwidth}{QQQl}
\lsptoprule
{Subdimension} & {Test} & {Conditions} & {Study}\\\midrule
\multicolumn{4}{c}{Affective dispositions}\\\midrule
Motivation

Foreign language learning anxiety

L2/L3 self-concepts & Student questionnaire & LAPS I: T1: 40 items (L2 French), T2: 72 items (L2 French\,+\,L3 English), LAPS II: T1: 40 items (L2 English), T2--3: 72 items (L2 English\,+\,L3 French) 4-point likert scale, no time constraints & LAPS I\&II\\
Locus of control & N-S personality scale (\citealt{NowickiStrickland1973}) & 20 yes/no items & LAPS I\&II\\\midrule
\multicolumn{4}{c}{Environmental factors}\\\midrule
SES Language background & Parental questionnaire & 20 items & LAPS I\&II\\\lspbottomrule
\end{tabularx}
\caption{Description of tests for affective dispositions and environmental factors}
\end{table}


\begin{table}\footnotesize
\begin{tabularx}{\textwidth}{QQQl}
\lsptoprule
{Subdimension} & {Test} & {Conditions} & {Study}\\\midrule
Fluid intelligence & CFT 20-R: Matrices \citep{Weiss2006} & 15 items, 3min & LAPS II\\
Fluid intelligence & CFT 20-R: Topological Deductions (Conditions) \citep{Weiss2006} & 11 items, 3min & LAPS II\\
Crystallised intelligence & CFT 20-R: Number Sequences \citep{Weiss2006} & 21 items, 12min & LAPS I\\
Visual working memory & Corsi Blocks & LAPS I: Start with 2 squares, 2 trials per level, 1 out of 2 trials must be correct to reach next level. LAPS II: Start with 2 squares, 3 trials per level, 1/3 trials must be correct to reach next level. & LAPS I\&II\\
Verbal working memory & Digit Span (Forward\slash Backward) & LAPS I: Start with 3 digits, 2 trials per level, 1/2 trials must be correct to reach next level. LAPS II: Start with 2 digits, 3 trials per level, 1/3 trials must be correct to reach next level. & LAPS I\&II\\
Automatic letter access, retrieval, and production & Alphabet Task \citep{BerningerEtAl1992} & Time limit: 60s, Scoring: number of legible letters in the correct alphabetic order in the first 15s & LAPS II\\
Creativity (divergent thinking) & Test of creative thinking (divergent production) (TCT-DP) \citep{UrbanJellen1995} & Maximum score: 72 no time constraints & LAPS I\\
Field independence & Group embedded figures test (GEFT) \citep{WitkinEtAl2014} & Part 1: 7 training items, 2min, not scored; Part 2: 9 test items, 5min, Part 3: 9 test items, 5min
Total score: 18 & LAPS I\&II\\
\lspbottomrule
\end{tabularx}
\caption{Description of tests for cognition/general learning abilities}
\end{table}

\begin{table}\footnotesize
\begin{tabularx}{\textwidth}{QQQl}
\lsptoprule
{Subdimension} & {Test} & {Conditions} & {Study}\\\midrule
School language German reading comprehension & ELFE 1--6, Reading Proficiency \citep{LenhardSchneider2006} & 72 words, 28 sentences, 20 short texts

Time limits:

LAPS I 4\textsuperscript{th}/5\textsuperscript{th} grade: 3/2min\textsubscript{word}, 3/2min\textsubscript{sentence}, 7/6min\textsubscript{text}

LAPS II reduced limits at T2 and T3 (cf. technical report) & LAPS I\&II\\
English 

listening and language use & Oxford Young Learners Placement Test

(\citealt{Testing2013}) & language use: 18 items

Listening: 12 items,

duration: no time constraints, approx. 35min & LAPS I (T2)\&II (T1)\\
General foreign language proficiency & C-tests French & 4 independent texts, 20 gaps per text.

Maximum score: 80

5min per text (20min) & LAPS I\\
General foreign language proficiency & C-tests English & 5 independent texts, 20 gaps per text.

Maximum score: 100

4min per text (20min) & LAPS II (T2, T3)\\
\lspbottomrule
\end{tabularx}
\caption{Description of language proficiency tests\label{tab:tests:part-4}}
\end{table}

\subsection{Piloting and adapting the test battery}\label{sec:02:3.6}

After piloting the test battery in LAPS I with 10 classes of 4\textsuperscript{th} and 5\textsuperscript{th} graders from nine different schools, the following changes were added in consultation with the scientific advisory board. The adapted version was used in LAPS II.

\subsubsection{L2 English proficiency measure}

For LAPS II, the dependent variable for L2 proficiency needed to be changed from L2 French to L2 English, due to curricular differences in the regions of LAPS I and LAPS II outlined in \sectref{sec:02:2.1}. This change had been anticipated and tests for L2 English had been selected at the start of the project.

For L2 English proficiency, we initially planned to use the Oxford Young Learners Placement Test (OYLPT, \citealt{Testing2013}). This is an online test assessing L2 English listening comprehension and language use (vocabulary and grammar) embedded in communicative situations. The OYLPT is supposed to cover CEFRL levels A0 to B1. The test seemed appropriate for two reasons: First, items are focused on communicative aspects of language use which is in keeping with curricular goals set for the target group. Second, because our participants are expected to reach A2 levels by the end of primary school, we assumed that the OYLPT was suitable to cover the range of proficiency levels in our sample. 

We were therefore surprised to find a large group of participants reaching close to the maximum score at the first time of testing (T1). To avoid ceiling effects at T2 and T3, we decided to change the English proficiency measure. In hindsight, it would have been important to pilot the OYLPT with a sample of learners comparable to our LAPS-II-learners as part of the trial phase. 

C-tests were chosen as an alternative because they have been shown to be a time-efficient and reliable measure of general language proficiency \citep{EckesGrotjahn2006}. Since the Swiss curriculum fosters all communicative skills, including reading and writing, participants have acquired the skills needed to cope with a C-tests. 

We modelled the C-tests on a version developed by \citet{WildenPorsch2017} for a similar target group of young learners in Germany. We also consulted C-tests for teenage learners of English by \citet{BabaiiShahri2010} to be able to capture higher levels of competence, i.e. to avoid ceiling effects. To make sure that the texts would be appropriate in terms of vocabulary knowledge, we consulted the English manuals used in the LAPS II region to identify content areas. We adapted the texts to include topics the children were likely to be familiar with. The C-tests were piloted with three classes (1\,×\,4\textsuperscript{th} grade, 2\,×\,5\textsuperscript{th} grades, 2\,×\,6\textsuperscript{th} grades) who did not participate in LAPS II. 

 \subsubsection{Other subdimensions}

Modifications to the test battery were also made for intelligence, phonetic coding ability, working memory, and creativity. 

In LAPS I, we used a measure of crystallized intelligence (number sequencing) to account for cognitive abilities unrelated to language. This was substituted with a test of fluid intelligence (matrices). Fluid intelligence was judged to be a more accurate assessment of general learning abilities, as it is independent of academic knowledge, such as reflected in the number sequencing test. We chose two subtests from a culture fair test (CFT 20-R, \citealt{Weiss2006}) with language-free and descriptive test items.

Originally, we assessed the aptitude subcomponent of phonetic coding ability with the LLAMA-E \citep{MearaEtAl2001}. The LLAMA-E is a measure of sound-symbol association and phonemic working memory. After discussions with our panel of experts, the sound-symbol aspect of this test was deemed too closely related to literacy skills, rather than the phonemic part of language aptitude which we intended to target. In order to have a more robust indication of the phonemic aptitude component, we therefore opted for the LLAMA-D subtest \citep{MearaEtAl2001} which measures phonemic discrimination and phonemic memory. 

We also decided to strengthen the working memory (WM) measure by complementing the forward digit span for verbal WM with a backward version, which some authors argue is also a measure of the central executive component (for an overview see e.g. \citealt{St-ClairEtAl2013}, \citealt{HilbertEtAl2014}). 

The Berninger-Graham Alphabet Task \citep{BerningerEtAl1992} was added as a speed test for automatic letter access, retrieval, and production. The Alphabet Task is easy to administer and has been found to be predictive of children’s levels of composition in the L1 (\citealt{BerningerEtAl1997}, \citealt{GrahamEtAl2006}). The test was chosen with regard to our aim to explore robust predictors for L2 proficiency. If the Alphabet Task turned out to be among them, it would be a convenient option for teachers wishing to assess their students’ L2 potential. To our knowledge, this possibility has not been explored previously.

We tried to accommodate these changes without adding to test taking time. We therefore decided to omit the TCT-DP for creative thinking from the test battery. This choice seemed justified, as creativity in connection with the task-based learning environment did not yield strong associations with learning outcomes in LAPS I (see Chapter 6). 

\section{Design}\label{sec:02:4}
\begin{sloppypar}
This section details participant characteristics, recruitment and procedures adopted in the LAPS I and LAPS II subprojects. In terms of the total number of participants, we draw attention to the fact that Tables~\ref{tab:02:2} and~\ref{tab:02:5} refer to the entire sample. The numbers reported in subsequent chapters may vary, e.g. due to the exclusion of students with L1 English or French. 

The test battery was administered by members of the LAPS team and/or trained research assistants. All instructions and time limits were recorded and played via speakers to have maximum control over the elicitation process, i.e. to create situations that were as similar as possible. 
\end{sloppypar}

 \subsection{Recruitment}


Participants for LAPS I and LAPS II were recruited via school administrators. Teachers decided voluntarily if they wanted to participate with their class. Written consent was obtained from the pupils’ parents. 

 \subsection{LAPS I}


In spring 2017, the test battery presented in Tables 2.1 to 2.4 was administered to 174 primary school pupils from 10 different classes from nine different schools (T1). In spring 2018, a second data collection (T2) took place in order to investigate students’ L3 (English) proficiency as well as French and English learning motivation (see \citealt{BertheleUdry2019} for an analysis of the skills in both FLs). Nine out of 10 classes from T1 participated in the second data collection. \tabref{tab:02:2} presents an overview of the participants of LAPS I.

Note that the variable for multilingualism is binary and based on the parent questionnaire administered at T1. Children being classified as multilingual met at least one of the following criteria:

\begin{itemize}
\item L1 other than German
\item Family language currently used: Other than German or German plus other
\item Literacy skills in a home language other than German
\item Child has received or is receiving tuition for non-native German speakers (referred to in the Swiss system as \textit{Deutsch als Zweitsprache, DaZ})
\item Child has received or is receiving tuition in the home language (referred to in Switzerland as \textit{Heimatliche Sprache und Kultur, HSK})
\end{itemize}

The number of L2 and L3 lessons is an estimated average of the total tuition received at the time of testing. The estimate is based on an evaluation of the cantons’ current timetables which was mandated by the government and conducted by \citet{BucherZemp2019}. According to the authors, the L2 is taught to Swiss primary school children with a total of 390 45-minute lessons. These lessons are distributed over 4 academic years (comprising 39 weeks each), usually with three weekly lessons in grades 3 and 4, and two weekly lessons in grades 5 and 6.

Depending on the canton, the number of L3 lessons taught at primary school ranges from 152 to 234.\footnote{In 2019, Appenzell Innerrhoden was the only canton to introduce L3 teaching at secondary school.} The total varies because of different timetables in grades 5 and 6, the period of L3 instruction: Some cantons opt for three weekly lessons per grade, while others offer only two weekly lessons per grade. 


\begin{table}
\fittable{\begin{tabular}{lccccccc} 
\lsptoprule
Time &	Grade &	$N$ & Multiling. & \multicolumn{2}{c}{Age} & \multicolumn{2}{c}{Mean $n$ lessons}\\\cmidrule(lr){5-6}\cmidrule(lr){7-8}
     &       &     &              &  Mean & Range & L2 French & L3 English\\\midrule
T1 & 4 & 57  & 11 & 10.3 & 9.4--11.0 &  172 & 0  \\
T2 & 5 & 55  & 10 & 11.3 & 10.5--11.6 & 248 & 58 \\
T1 & 5 & 117 & 27 & 11.5 & 10.6--13.2 & 248 & 58 \\
T2 & 6 & 103 & 23 & 12.6 & 10.5--14.2 & 324 & 134\\
T1 &   & 174 & 38 & 11.1 & 9.4--13.2  &     &    \\
T2 &   & 158 & 33 & 12.1 & 10.5--14.2 &     &    \\
\lspbottomrule
\end{tabular}}
\caption{Participants LAPS I\label{tab:02:2}}
\end{table}

  \subsection{LAPS I procedure T1}


Data collection took place between March and April 2017. Tests were administered by two or three assistants or LAPS-researchers, depending on the number of students per class. The administration of the entire test battery took approximately 3 hours and was divided into two sessions in order to prevent fatigue scheduled within a week (see \tabref{tab:02:3}). The test sequence was organized to allow for alternation between more and less cognitively demanding tasks. For practical reasons the order of the tasks could not be varied. It is possible that there are order effects of the tasks. However, if so, they arguably affect the participants in a similar way. Since we are interested in differences among the students, we deem the impact of such order effects on the overall results to be modest.


\begin{table}
\begin{tabularx}{\textwidth}{llQ}
\lsptoprule

Session 1 (${\approx}$100min) & Slot 1 & Introduction; Language Detective; GEFT\\
                              & Slot 2 & Team up Words!; ELFE 1--6\\
Session 2 (${\approx}$90min) & Slot 3 & C-test French; TCT-DP; CFT 20-R (numerical sequences)\\
                             & Slot 4 & LLAMA-E; Corsi Blocks; Digit Span; Motivation Questionnaire\\
\lspbottomrule
\end{tabularx}
\caption{Procedure LAPS I, T1\label{tab:02:3}}
\end{table}

\subsection{LAPS I procedure T2}


Data collection took place a year later in April and May 2018. We administered measures of L3 English proficiency and motivation in one session that lasted approximately 75 minutes. This session was conducted by two assistants or LAPS researchers, applying standardized instructions and procedures. Language tests and questionnaires were alternated (see \tabref{tab:02:4}). 

\begin{table}
\begin{tabularx}{\textwidth}{lQ}
\lsptoprule
Test session\footnote{(Short breaks between tasks)} & Introduction; Questionnaire L2 English; Oxford Young Learners Placement Test; Questionnaire L3 French\\
\lspbottomrule
\end{tabularx}
\caption{Procedure LAPS I T2 – Spring 2018\label{tab:02:4}}
\end{table}

 \subsection{LAPS II}\label{sec:02:4.5}


After LAPS I, minor changes discussed in section 3 were made to the test battery. In autumn 2017 (T1), the adapted version was administered to primary school pupils from 32 different classes (13 4\textsuperscript{th} graders, 15 5\textsuperscript{th} graders and 4 mixed grade classes). As in LAPS I, instructions and procedures were standardized and introduced to research assistants in a training session. 

In spring 2018 (T2) and 2019 (T3) five measures were re-administered to the same participants to monitor longitudinal development: L2 English proficiency, English/French motivation questionnaire, language of instruction German, grammatical sensitivity, and inductive ability. A total of 637 pupils participated in LAPS~II. \tabref{tab:02:5} provides participant details. Note that the same criteria apply to the variable \textit{multilingual} as in LAPS I. 


\begin{table}
\fittable{\begin{tabular}{lccccccc} 
\lsptoprule
Time &	Grade &	$N$ & Multiling. & \multicolumn{2}{c}{Age} & \multicolumn{2}{c}{Mean $n$ lessons}\\\cmidrule(lr){5-6}\cmidrule(lr){7-8}
     &        &     &              &  Mean & Range & L3 French & L2 English\\\midrule
\multicolumn{8}{c}{4\textsuperscript{th} graders at T1}\\\midrule
T1 & 4 & 289 & 171 & 10.0 & 9.1--11.6  &    & 181\\
T2 & 4 & 274 & 161 & 10.6 & 9.7--12.2  &    & 259\\
T3 & 5 & 260 & 150 & 11.6 & 10.7--13.2 & 66 & 346\\\midrule
\multicolumn{8}{c}{5\textsuperscript{th} graders at T1}\\\midrule
T1 & 5 & 326 & 157 & 11.0 & 9.2--12.4 & 14 & 294\\
T2 & 5 & 304 & 143 & 11.6 & 9.8--13.0 & 66 & 346\\
T3 & 6 & 306 & 146 & 12.6 & 10.8--14.0 & 146 & 426\\\midrule
\multicolumn{8}{c}{Total participants}\\\midrule
T1 & 615 & 328 & 10.5 & 9.1--12.4 &   & \\
T2 & 578 & 304 & 11.1 & 9.7--13.0 &   & \\
T3 & 566 & 296 & 12.1 & 10.7--14.0&   & \\
\lspbottomrule
\end{tabular}}
\caption{Participants LAPS II. T1: Autumn 2017; T2: Spring 2018; T3: Spring 2019\label{tab:02:5}}
\end{table}

 \subsection{LAPS II procedure T1}

The administration of the entire test battery took approximately 3.5 hours. Similar to LAPS I, it was divided into two sessions (see \tabref{tab:02:6}). Nine assistants were recruited and trained for test administration.

\begin{table}
\begin{tabularx}{\textwidth}{llQ}
\lsptoprule
Session 1 (${\approx}$120min) & Slot 1 & Introduction; Language Detective; GEFT\\
& Slot 2 & ELFE 1--6; CFT 20-R\\
& Slot 3 & Alphabet Task; Team up Words!\\
Session 2 (${\approx}$90min) & Group A 2 A & LLAMA-D; Corsi blocks; Digit Span (f/b); Motivation Questionnaire English\\
& Group B 1A+ teacher & Oxford Young Learners Placement Test (OYLPT); Motivation Questionnaire English\\
\lspbottomrule
\end{tabularx}
\caption{Procedure LAPS II T1: Autumn 2017\label{tab:02:6}}
\end{table}

 \subsection{LAPS II procedure T2 and T3}


At T2 and T3, selected measures from the test battery were repeated to longitudinally track their development: L2 English proficiency, language of instruction German, motivation, grammatical sensitivity, and inductive ability. They were administered in one session of approximately 90 minutes divided into two slots (\tabref{tab:02:7}). As only paper and pencil group tests were administered at T2 and T3, fewer assistants were needed for data collection. Four research assistants at T2 and two research assistants at T3 were recruited and trained analogous to T1.


\begin{table}
\begin{tabularx}{\textwidth}{llQ}
\lsptoprule
Test session 1--2TL & Slot 1 & Introduction; Language Detective\\
                   &  & Team up Words!; English Motivation Questionnaire\\
                   & Slot 2 & ELFE 1--6; English Proficiency (C-test); French Motivation Questionnaire\\
\lspbottomrule
\end{tabularx}
\caption{Procedure LAPS II T2 and T3: Spring 2018 \& 2019\label{tab:02:7}}
\end{table}

\section{Scoring and data entry}

In LAPS I and LAPS II, computer-administered tests were scored automatically. Paper and pencil tasks were scored by members of the LAPS team and assistants who participated in the data collection. Subsequently, all data were entered and stored for analysis. To minimise the chances of mishaps in data entry, a special platform was created restricting the data format of the input and displaying error messages when impossible data was entered (cf. \citealt{Vanhove2018}). 

Test scoring and data entry procedures were defined in a manual and the research assistants were trained accordingly. The most important aspects of the process will be summarised in the following paragraphs. Full details on handling missing values and scoring the tests are given in the technical report (\url{https://osf.io/d9gnh/}).

For the English proficiency tests administered at T2 and T3, two different scoring types were applied: 

\begin{itemize}
\item
Spelling errors are penalized: This constitutes the standard scoring type.
\item
Spelling errors are ignored: This accounts for the communicative language learning approach practised in Swiss EFL-classrooms which doesn’t focus strongly on spelling.
\end{itemize}

Due to the acceptance of phonetically correct spellings, the second scoring type opens the door to variable criteria as to what represents an acceptable response and what not. When trialling the C-tests, lists of accepted and unaccepted spellings were defined. At T2 and T3, a randomly selected subset was scored by two independent raters and the guidelines for spelling variants were supplemented accordingly. 

A subsequent analysis reveals that the two scoring types are strongly correlated ($r_{\text{T2}}=0.98, r_{\text{T3}}=0.98$, cf. technical report, Figure 16.1 and Figure 17.1). For any statistical analysis we used the first scoring type, i.e. where spelling errors are penalized.
\sloppy\printbibliography[heading=subbibliography,notkeyword=this]
\end{document}
